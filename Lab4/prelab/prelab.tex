\documentclass[oneside, 10pt]{article}
\usepackage[margin=1.2in]{geometry}
\usepackage[english]{babel}
\usepackage{amsfonts}
\usepackage{amsmath}
\usepackage{amssymb}
\usepackage{minted}
\usepackage[utf8]{inputenc}
\usepackage[T1]{fontenc}
\usepackage{listings}
\usepackage{subfig}
\usepackage[pdfencoding=auto, psdextra]{hyperref}
\usepackage[
    type={CC},
    modifier={by-nc-sa},
    version={4.0},
]{doclicense}

\title{\LARGE CSC258 PRELAB \#4}
\author{\href{https://tingfengx.github.io}{Tingfeng Xia}}
\date{\today}
\begin{document}
\maketitle
\section*{PART I}
\paragraph{1.} Here is my logic gate level schematic
\begin{center}

\end{center}
\paragraph{4.} To avoid uncertainty, we shall avoid any case where $Clk\gets 0$ 
at initial state. Since $D$ is unspecified, the behavior of the circuit can be 
unpredictable. 

\section*{PART II}
\paragraph{1.} Here is my code for RegisterALU:

\paragraph{2.} Here are the screen shots for the simulations:

\section*{PART III}
\paragraph{1.} If \texttt{load\_n = 1} and \texttt{ShiftRight = 0}, then the 
register remains unchanged during the entire process. Since \texttt{ShiftRight}
is connected to the \texttt{shift} input of each ShifterBit and this essentially 
feed back the register with its own value. 
\end{document}
